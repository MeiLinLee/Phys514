\documentclass{article}


\usepackage{NotesStyle}

% Cover info

\title{Phys 514 \\
	\large Problem Set 2}

\author{April Sada Solomon - 260708051}
\date{Winter 2021}

\begin{document}
	\maketitle
	\newpage
	\tableofcontents
	\newpage
	
	\pagenumbering{arabic}
	\setcounter{page}{1}
	\cfoot{\thepage}
	
	\section{A bit of calculus}
		Consider $\R^3$ in the Cartesian coordinates $x^i = (x,y,z)$. Hmmmm... Wait a minute, something's not right. Ah, there we go. Consider $\R^3$ in the Cartesian coordinates $\ex^i = (\ex,\ey,\ez)$ with spherical coordinates $\ex^{i'} = (\er, \etheta, \ephi)$ as defined\footnote{This document is written in \LaTeX, $\, $so clearly using emojis is true and validated.} by
		\begin{align*}
			\ex &= \er \sin \etheta \cos \ephi \\
			\ey &= \er \sin \etheta \sin \ephi \\
			\ez &= \er \cos \etheta
		\end{align*}
		\subsection{Particle's path}
			Consider the world line $\ex^i (\elambda) = (\cos \elambda, \sin \elambda, \elambda)$. We want to express this world line in the spherical coordinates as defined above such that
			
			$$ \ex^i = \begin{pmatrix}
				\ex \\
				\ey \\
				\ez
			\end{pmatrix} = \begin{pmatrix}
			\cos \elambda \\
			\sin \elambda \\
			\elambda
			\end{pmatrix}$$
			If we wish to find a parametrization such that $(\ex, \ey, \ez) \to (\er, \etheta, \ephi)$ then it follows that
			\begin{align*}
				\er^2 &= \ex^2 + \ey^2 + \ez^2 \\
				\etheta &= \arctan \left( \frac\ey\ex \right) \\
				\ephi &= \arctan \left( \frac{\sqrt{\ex^2 + \ey^2}}{\ez} \right)
			\end{align*}
			So $\ex^2 + \ey^2 = \cos^2 \elambda + \sin^2 \elambda = 1$ implies that 
			\begin{align*}
				\er &= \pm \sqrt{\elambda^2 +1} \\
				\etheta &= \arctan \left( \frac{\sin \elambda}{\cos \elambda} \right) = \elambda \\
				\ephi &= \arctan \left( \frac{1}{\elambda} \right) = \cot^{-1} \elambda
			\end{align*}
			Hence, $\ex^i (\elambda) = (\sqrt{\elambda^2 + 1}, \elambda, \cot^{-1} \elambda)$.
			
		\subsection{Tangential Vector}
			Recall the tangential vector $T^i (\elambda) = \frac{d\ex^i}{d\elambda}$. Thus,
			$$ T^i (\elambda) = \frac{d}{d\elambda} \begin{pmatrix}
				\cos \elambda \\
				\sin \elambda \\
				\elambda
			\end{pmatrix} = \begin{pmatrix}
			-\sin \elambda \\
			\cos \elambda \\
			1
			\end{pmatrix}$$
			The Chain Rule becomes extremely useful when considering the change of coordinates $\ex^{i}$ such that
			$$ \bar{T}^{i} (\elambda) = \frac{d\bar{\ex}^{i}}{d \elambda} = \frac{d \bar{\ex}^{i}}{d \ex^j} \frac{d \ex^j}{d \et}$$
			where $\et$ represents time. Using the result from 1.a we get
			$$ \bar{T}^{i} (\elambda) = \frac{d}{d\elambda} \begin{pmatrix}
				\sqrt{\elambda^2 + 1} \\
				\elambda \\
				\cot^{-1} \elambda 
			\end{pmatrix} = 
			\begin{pmatrix}
				\frac{\elambda}{\sqrt{\elambda^2 + 1}} \\ 
				1 \\
				-\frac{1}{\elambda^2 + 1}
			\end{pmatrix}$$
		 	Spherical coordinates is easy in these cases. But what if we were the type of person that uses Arch Linux? Then we obviously would want to find the Cartesian coordinates:
		 	
		 	\begin{align*}
		 		\bullet \quad \bar{T}^{1} &= \frac{\partial \left( \sqrt{\ex^2 + \ey^2 + \ez^2} \right)}{\partial \ex} \left( -\sin \elambda \right) + \frac{\partial \left( \sqrt{\ex^2 + \ey^2 + \ez^2} \right)}{\partial \ey} \left( \cos \elambda \right) \\
		 		&\longspace + \frac{\partial \left( \sqrt{\ex^2 + \ey^2 + \ez^2} \right)}{\partial \ez} \left( 1 \right) \\
		 		&= \frac{\ex}{\er} \left( -\sin \elambda \right) + \frac{\ey}{\er} \left(\cos \elambda \right) + \frac{\ez}{\er} (1) \\
		 		&= - \frac{\cos \elambda \sin \elambda}{\er} + \frac{\cos\elambda \sin \elambda}{\er} + \frac{\ez}{\er} \\
		 		&= \boxed{\frac{\elambda}{\sqrt{\elambda^2 + 1}} }\\
		 		\bullet \quad \bar{T}^2 &= \frac{\partial \arctan \left( \frac\ey\ez \right)}{\partial \ex}(-\sin \elambda) + \frac{\partial \arctan \left( \frac\ey\ez \right)}{\partial \ey} \cos \elambda + \frac{\partial \arctan \left( \frac\ey\ez \right)}{\partial \ez} (1) \\
		 		&= \frac{\ey}{\ex^2 + \ey^2} \sin \elambda + \frac{\ex}{\ex^2 + \ey^2} \cos \elambda \\
		 		&= \boxed{ 1 }
		 	\end{align*}
	 		\begin{align*}
	 			\bullet \quad \bar{T}^3 &= \frac{\partial \arctan \left( \frac{\sqrt{\ex^2 +\ey^2}}{\ez} \right)}{\partial \ex} (-\sin \elambda) + \frac{\partial \arctan \left( \frac{\sqrt{\ex^2 +\ey^2}}{\ez} \right)}{\partial \ey} (\cos \elambda) \\
	 			&\longspace + \frac{\partial \arctan \left( \frac{\sqrt{\ex^2 +\ey^2}}{\ez} \right)}{\partial \ez} (1) \\
	 			&= -\frac{\ex \ez \sin \elambda}{\sqrt{\ex^2 + \ey^2} \er^2} +\frac{\ey \ez \cos \elambda}{\sqrt{\ex^2 + \ey^2} \er^2} - \frac{\sqrt{\ex^2 + \ey^2}}{\er^2} \\
	 			&= \frac{\ez}{\sqrt{\ex^2 + \ey^2} \er} \left( - \cos \elambda \sin \elambda + \sin \elambda \cos \elambda \right) - \frac{\sqrt{\ex^2 + \ey^2}}{\er} \\
	 			&= \boxed{-\frac{1}{\elambda^2 + 1}} 
	 		\end{align*}
 			Therefore:
 			\begin{align*}
 				T^i (\elambda) &= \begin{pmatrix}
 					\cos \elambda \\
 					\sin \elambda \\
 					\elambda
 				\end{pmatrix} &
 				\bar{T}^i (\elambda) &= \begin{pmatrix}
 					\frac{\elambda}{\sqrt{\elambda^2 + 1}} \\ 
 					1 \\
 					-\frac{1}{\elambda^2 + 1}
 					\end{pmatrix}
 			\end{align*}
 		\pagebreak
 		\section{Vector fields}
 			\subsection{Infinitesimal vector translations}
 			Alright, no more emojis for this assignment $\ex$. Let $\dot{x} = \dot{x}^\mu d\mu = \frac{\partial \alpha^\mu}{\partial \lambda}\partial_\mu$ where $\partial_\mu = (\partial_1, \partial_2, \partial_3)$. Thus, it follows that
 			$$ x^\mu \to \alpha^\mu (x^\mu, \lambda)$$ 
 			Now consider the transformation
 			$$ T = I + \lambda \dot{x}$$
 			Then we have
 			\begin{align*}
 				\alpha &= T x \\
 				&= (I + \lambda \dot{x}) x \\
 				&= \left( I + \lambda \frac{\partial \alpha^\mu}{\partial \lambda \frac{\partial}{\partial x^\mu}} \right)x \\
 				&= x + \lambda \frac{\partial \alpha^\mu}{\partial \lambda \frac{\partial x}{\partial x^\mu}} \\
 				&= x + \lambda \frac{\partial \alpha^\mu}{\partial \lambda} \\
 				&= x + \lambda \dot{x}^\mu
 			\end{align*} 
 			As expected, we have $\alpha^\mu = x^\mu + \lambda \dot{x}^\mu$. We can now change coordinates such that
 			\begin{align*}
 				\alpha^i &= x^i + \lambda \frac{\partial \alpha^i}{\partial \lambda} \frac{\partial x^i}{\partial x^\mu} \\
 				\intertext{where we have $\partial_i = \frac{\partial}{\partial x^i} $ for $i=1,2,3$. Thus:}
 				\alpha^i &= x^i + \lambda \frac{\partial \alpha^i}{\partial \lambda} (x^1, x^2, x^3)
 				\intertext{where $i \neq 1 \lor 2 \lor 3 \implies x^1 \lor x^2 \lor x^3 = 0$, and $i = 1 \lor 2 \lor 3 \implies x^1 \lor x^2 \lor x^3 = x^i$. So for $i=1$ we have $\frac{\partial x^1}{\partial x^1} = (x^1, 0,0)$, for $i=2$ we have $\frac{\partial x^i}{\partial x^i} = (x^1, 0,0)$, and for $i=3$ we have $(0,0,x^3)$. Hence, considering $x^1$:}
 				\alpha^1 &= x^1 + \lambda \frac{\partial \alpha^i}{\partial \lambda} (x^1, 0, 0) \\
 				&=    x^1 \left( 1 + \lambda \frac{\partial \alpha^1}{\partial \lambda} \right)
 			\end{align*}
 			As $\lambda \to 0$, it is clear that $\alpha^1$ represents an infinitesimal displacement of $x^1$ on the $x^1$ axis. It follows similarly for $i=2$ and $i=3$.
 			
 			\subsection{Infinitesimal vector rotations}
 			Given
 			$$
 			\alpha^\mu = 
 			\begin{pmatrix}
 				v\\
 				w\\
 				z
 			\end{pmatrix} =
 			\begin{pmatrix}
 				x^2 \partial_3   - x^3 \partial_2 \\
 				x^3 \partial_1  - x^1 \partial_3 \\
 				x^1 \partial_2  - x^2 \partial_1 \\
 			\end{pmatrix}
			$$
			we have the following Jacobians for each vector respectively: 
			\begin{align*}
				J_{v} &= 
					\begin{pmatrix}
						0		&		0		&		0		\\
						0		&	0		&	-1		\\
						0		&	1	&	0
					\end{pmatrix},
					 &
				J_{w}  &= 
					\begin{pmatrix}
						0	&		0		&	1		\\
						0		&		0		&	0			\\
						-1	&		0		&	0
					\end{pmatrix}, &
				J_z &= 
					\begin{pmatrix}
						0	&	-1		&		0		\\
						1		&	0		&	0		\\
						0		&	0		&	0
					\end{pmatrix}
			\end{align*}
			for $x^{\mu'} = (I + \lambda J)x^\mu$. Therefore, the rotations with respect to each vector will be given by
			\begin{align*}
				T_{1,\mu} &= x^\mu + \lambda v = \begin{pmatrix}
					x^1 \\
					x^2 - \lambda x^3 \\
					x^3 + \lambda x^2
				\end{pmatrix}, &
				T_{2,\mu} &= x^\mu + \lambda w = \begin{pmatrix}
					x^1 + \lambda x^3 \\
					x^2  \\
					x^3 - \lambda x^1
				\end{pmatrix}, &
				T_{3,\mu} &= x^\mu + \lambda v = \begin{pmatrix}
					x^1 - \lambda x^2\\
					x^2 + \lambda x^1 \\
					x^3
				\end{pmatrix} \\
			\end{align*}
 			For the limit of $\lambda \to 0$ as $T_\mu^\nu$ forms an infinitesimal rotation about $x^\nu$, as desired.
 			
 		\subsection{Radial coordinate $r$}
 			It is clear that
 			$$ r^2 = (x^1)^2 + (x^2)^2 + (x^3)^2$$
 			So the dot products are given by:
 			\begin{itemize}
 				\item $ $\vspace{-1cm}
 					\begin{align*}
 						vr &= x^2 \frac{\partial}{\partial x^3} \sqrt{(x^1)^2 + (x^2)^2 + (x^3)^2} - x^3 \frac{\partial}{\partial x^2} \sqrt{(x^1)^2 + (x^2)^2 + (x^3)^2} \\
 						&= \frac{x^2 x^3}{r} - \frac{x^3 x^2}{r} = 0
 					\end{align*}
 				\item $ $\vspace{-1cm}
 					\begin{align*}
 						wr &= x^3 \frac{\partial}{\partial x^1} \sqrt{(x^1)^2 + (x^2)^2 + (x^3)^2} - x^1 \frac{\partial}{\partial x^3} \sqrt{(x^1)^2 + (x^2)^2 + (x^3)^2} \\
 						&= \frac{x^3 x^1}{r} - \frac{x^1 x^3}{r} = 0
 					\end{align*}
 				\item $ $\vspace{-1cm}
	 				\begin{align*}
	 					zr &= x^1 \frac{\partial}{\partial x^2} \sqrt{(x^1)^2 + (x^2)^2 + (x^3)^2} - x^2 \frac{\partial}{\partial x^1} \sqrt{(x^1)^2 + (x^2)^2 + (x^3)^2} \\
	 					&= \frac{x^1 x^2}{r} - \frac{x^2 x^1}{r} = 0
	 				\end{align*}
 			\end{itemize}
 			As we saw in the rest of Problem 2, $x^\mu \times \partial_\mu$ represents a rotation about an axis in the coordinate system. As rotations change the direction of a vector but not its magnitude, we conclude that:
 			$$ (x^\mu \times \partial_\mu)r = \vec{0}$$
 			\subsection{Spherical $z$ vector components}
 			We have
 			$ z = x^1 \partial_2 - x^2 \partial_1 = (-x^2, x^1, 0)$
 			where $\partial_\mu = \frac{\partial}{\partial x^\mu}$. We want to find spherical coordinates $x^{\mu'}$. By Chain Rule, we know that
 			$$ \partial_\mu = \frac{\partial}{\partial x^\mu} = \frac{\partial x^{\mu '}}{\partial x^\mu} \frac{\partial}{\partial x^{\mu'}} = J \partial_{\mu '}$$
 			$$ \therefore \partial_{\mu'} = J^{-1} \partial_\mu = \frac{\partial x^\mu}{\partial x^{\mu '}} \partial_\mu$$
 			Thus, it follows that:
 			\begin{align*}
 				z(x^\mu) &= x^1 \partial_2 - x^2 \partial_1 \\
 				\therefore z(x^{\mu'}) &= x^1(x^{\mu'}) \frac{\partial x^{\mu'}}{\partial x^2} \frac{\partial}{\partial x^{\mu'}} - x^2 (x^{\mu '}) \frac{\partial x^{\mu'}}{\partial x^1}\frac{\partial }{\partial x^{\mu'}}
 			\end{align*}
 			So the inverse Jacobian $J^{-1} = \frac{\partial x^{\mu'}}{\partial x^\mu}$ is
 			$$\frac{\partial x^{\mu'}}{\partial x^\mu} = 
 				\begin{pmatrix}
 					\frac{\partial r}{\partial x}	&	\frac{\partial r}{\partial y} &	\frac{\partial r}{\partial z} \\
 					\frac{\partial \theta}{\partial x} & \frac{\partial \theta}{\partial y} & \frac{\partial \theta}{\partial z} \\
 					\frac{\partial \phi}{\partial x} & \frac{\partial \phi}{\partial y} & \frac{\partial \phi}{\partial z} 				
 				\end{pmatrix} = \begin{pmatrix}
 				2x & 2y & 2z \\
 				\frac{xz}{r\sin \theta} & \frac{yz}{r\sin \theta} & -r \sin \theta \\
 				- \frac{y}{\left(r \sin \theta \right)^2} & \frac{x}{\left(r \sin \theta \right)^2} & 0
 			\end{pmatrix}
 			$$
 			for $\frac{\partial}{\partial x^{\mu'}} = \left( \frac{\partial}{\partial r}, \frac{\partial}{\partial \theta}, \frac{\partial}{\partial \phi}\right)$, $x^1 (x^{\mu'}) = r\sin\theta\cos\phi$, and $x^2 (x^{\mu'}) = r\sin\theta\sin\phi$.
 			Hence,
 			\begin{align*}
 				\frac{\partial x^{\mu'}}{\partial x^1} \frac{\partial}{\partial x^{\mu'}} &= \begin{pmatrix}
 					2x \frac{\partial}{\partial r} \\
 					\frac{xz}{r\sin\theta}\frac{\partial}{\partial \theta} \\
 					\frac{-y}{(r\sin\theta)^2}\frac{\partial}{\partial \phi}
 				\end{pmatrix} & 
 				\frac{\partial x^{\mu'}}{\partial x^2} \frac{\partial}{\partial x^{\mu'}} &= \begin{pmatrix}
 					2y \frac{\partial}{\partial r} \\
 					\frac{yz}{r\sin\theta}\frac{\partial}{\partial \theta} \\
 					\frac{x}{(r\sin\theta)^2}\frac{\partial}{\partial \phi}
 				\end{pmatrix}
 			\end{align*}
 			$$ \therefore  z(x^{\mu'}) = x^1(x^{\mu'}) \frac{\partial x^{\mu'}}{\partial x^2} \frac{\partial}{\partial x^{\mu'}} - x^2 (x^{\mu '}) \frac{\partial x^{\mu'}}{\partial x^1}\frac{\partial }{\partial x^{\mu'}} = r\sin\theta\cos\phi \begin{pmatrix}
 				2y \frac{\partial}{\partial r} \\
 				\frac{yz}{r\sin\theta}\frac{\partial}{\partial \theta} \\
 				\frac{x}{(r\sin\theta)^2}\frac{\partial}{\partial \phi}
 			\end{pmatrix} - r\sin\theta\sin\phi \begin{pmatrix}
	 			2y \frac{\partial}{\partial r} \\
	 			\frac{yz}{r\sin\theta}\frac{\partial}{\partial \theta} \\
	 			\frac{x}{(r\sin\theta)^2}\frac{\partial}{\partial \phi}
 			\end{pmatrix}$$
 			Now, notice that $(x^1)^2 + (x^2)^2 = r^2 \sin\theta$ and
 			$$ 
 			\begin{pmatrix}
 				x^1 \\
 				x^2 \\
 				x^3
 			\end{pmatrix} = \begin{pmatrix}
 			r\sin\theta\cos\phi \\
 			r\sin\theta\sin\phi \\
 			r\cos\theta
 		\end{pmatrix}$$
 		So we can now finally compute each component of the $z$ vector in spherical coordinates:
 		\begin{align*}
 			x^{1'} &= &2xy \partial r - 2yx \partial r& &= 0 \\
 			x^{2'} &= &\frac{xyz}{\sqrt{x^2 + y^2}} \partial \theta - \frac{yxz}{\sqrt{x^2 + y^2}} \partial \theta& &= 0\\
 			x^{3'} &= &\frac{x^2}{x^2 + y^2} \partial \phi + \frac{y^2}{x^2 + y^2} \partial \phi& &= \partial \phi
 		\end{align*}
 		$$ \therefore \boxed{z = \begin{psmallmatrix}
 			0\\
 			0\\
 			\partial \phi
 		\end{psmallmatrix}}$$
 		\subsection{Lorentz boost}
 		From the Carroll book, we have the following definition of a Lorentz boost over Minkowski spacetime:
 		
 		$$ \Lambda_\nu^{\mu'} = \begin{pmatrix}
 			\cosh \phi	&	- \sinh \phi	&	0	&	0	\\
 			-\sinh \phi & \cosh \phi &	0	&	0	\\
 			0	&	0	&	1	&	0 \\
 			0	&	0	&	0	&	1
 		\end{pmatrix}$$
 		such that $x' = \Lambda x$. Therefore, we define $\M^2$ spacetime $(t,x)$ such that
 		$$ \Lambda = \begin{pmatrix}
 			\cosh \lambda & - \sinh \lambda \\
 			- \sinh\lambda & \cosh \lambda
 		\end{pmatrix} $$
 		where $\lambda$ is the parametrization of the boost over a worldline in the $x$ direction. Hence:
 		
 		\begin{align*}
 			\begin{pmatrix}
 				t' \\
 				x' 
 			\end{pmatrix} & =
 			\begin{pmatrix}
 				\cosh \lambda & - \sinh \lambda \\
 				- \sinh\lambda & \cosh \lambda
 			\end{pmatrix}
 			\begin{pmatrix}
 				t \\
 				x 
 			\end{pmatrix} \\
 			&= \begin{pmatrix}
 				t \\
 				x 
 			\end{pmatrix}\cosh \lambda - \begin{pmatrix}
 				x \\
 				t 
 			\end{pmatrix}\sinh \lambda 
 		\end{align*}
 		Now assume that $\lambda$ is infinitesimally small such that $\cosh \lambda \to 1$ and $\sinh \lambda \to 2$. Then, it follows that
 		$$ 
 			\begin{pmatrix}
 				t' \\
 				x' 
 			\end{pmatrix} = 
 			\begin{pmatrix}
 				t \\
 				x 
 			\end{pmatrix} - \lambda \begin{pmatrix}
 			x \\
 			t 
 			\end{pmatrix}
 		$$
 		$$ \therefore x' = x-\lambda v$$
 		The velocity $v$ is given by
 		$$ v = \begin{pmatrix}
 			x \\ t
 		\end{pmatrix} = x \partial_t + t \partial_x$$ where for infinitesimally small $\lambda$ we have $\partial_t \to (1,0)$ and $\partial_x \to (0,1)$. Therefore:
 		\begin{align*}
 			x \partial_t &\to (x,0) \\
 			t \partial_x &\to (0,t)
 		\end{align*}
 		So $$\boxed{v = x \partial_t + t \partial_x = \begin{pmatrix} x \\ t \end{pmatrix}}$$
 		is the vector yielding a boost on the $x$ direction, with a Lorentzian matrix given by:
 		$$ \Lambda = \begin{pmatrix}
 			1 & -\lambda \\
 			-\lambda & 1
 		\end{pmatrix}$$
 		
 \pagebreak	
 	\section{Lie algebra}
 		\subsection{Vector transform}
	 		We have the commutator:
	 		$$ \left[ \ex, \ey \right]^{\er} = \ex^{\ez} \partial_{\ez} \ey^{\er} - \ey^{\ez} \partial_{\ez} \ex^{\er}$$
	 		SIKE, just kidding. Imagine doing GR with emojis tho. Einstein would be so proud of us.
	 		$$ [v,w]^\nu = v^\mu \partial_\mu w^\nu - w^\mu \partial_\mu v^\nu$$
	 		Now we let $\lambda$ be a smooth function on the manifold on which the vectors are defined such that
	 		\begin{align*}
	 			[v,w]^\nu (\lambda) &= \left[ v^\mu \partial_\mu, w^\nu \partial_\nu \right] (\lambda) \\
	 			&= v^\mu \frac{\partial}{\partial x^\mu} \left( w^\nu \frac{\partial \lambda}{\partial x^\nu}\right) - w^\mu \frac{\partial}{\partial x^\mu} \left( v^\nu \frac{\partial \lambda}{\partial x^\nu}\right) \\
	 			\intertext{By the product rule, we get:}
	 			&= v^\mu \frac{\partial w^\nu}{\partial x^\mu} \frac{\partial \lambda}{\partial x^\nu} + v^\nu w^\nu \frac{\partial^2 \lambda}{\partial x^\mu \partial x^\nu} - w^\mu \frac{\partial v^\nu}{\partial x^\mu} \frac{\partial \lambda}{\partial x^\nu} - w^\mu v^\mu \frac{\partial^2 \lambda}{\partial x^\nu \partial x^\mu} \\
	 			&= v^\mu \frac{\partial w^\nu}{\partial x^\mu} \frac{\partial \lambda}{\partial x^\nu} - w^\mu \frac{\partial v^\nu}{\partial x^\mu} \frac{\partial \lambda}{\partial x^\nu} \\
	 			&= \left( v^\mu \frac{\partial w^\nu}{\partial x^\mu} \frac{\partial }{\partial x^\nu} - w^\mu \frac{\partial v^\nu}{\partial x^\mu} \frac{\partial}{\partial x^\nu} \right) (\lambda)\\
	 			&= \left( v^\mu \partial_\mu w^\nu - w^\mu \partial_\mu v^\nu \right) (\lambda)
	 		\end{align*}
 			So therefore,
 			$$ [v,w]^\nu = \left[ v^\mu \partial_\mu, w^\nu \partial_\nu \right] =  v^\mu \frac{\partial w^\nu}{\partial x^\mu} \frac{\partial }{\partial x^\nu} - w^\mu \frac{\partial v^\nu}{\partial x^\mu} \frac{\partial}{\partial x^\nu}$$
 			As needed.
 			\pagebreak
 		\subsection{Commutator computation}
 			We have the commutator:
 			$$ \left[ v,w\right]^\mu = v^\mu \partial_\mu w^\nu - \omega^\mu \partial_\mu v^\nu = \sum_i \sum_j \left( v^j \partial_j w^i - w^j \partial_j v^i \right) \partial_i$$
 			Therefore:
 			$$ v = \left[0, -x^3, x^2 \right]^T, \quad\quad w = \left[x^3, 0, -x^1 \right]^T, \quad \quad z = \left[-x^2, x^1, 0 \right]^T$$
 			which leads us to the following Jacobians:
 		\begin{align*}
 			J_{v} &= 
 			\begin{bsmallmatrix}
 				0		&		0		&		0		\\
 				0		&	0		&	-1		\\
 				0		&	1	&	0
 			\end{bsmallmatrix},
 			&
 			J_{w}  &= 
 			\begin{bsmallmatrix}
 				0	&		0		&	1		\\
 				0		&		0		&	0			\\
 				-1	&		0		&	0
 			\end{bsmallmatrix}, &
 			J_z &= 
 			\begin{bsmallmatrix}
 				0	&	-1		&		0		\\
 				1		&	0		&	0		\\
 				0		&	0		&	0
 			\end{bsmallmatrix}
 		\end{align*}
 			Now recall the definition of a commutator via Jacobian:
 			$$ [X,Y] := J_y X - J_x Y$$
 			Therefore:\vspace{-1cm}
 			\begin{align*}
 				[v,w] &= J_w v - J_v w \\
 				&= \begin{bsmallmatrix}
 					0	&	0	&	1 \\
 					0	&	0	&	0 \\
 					-1	&	0	&	0	
 				\end{bsmallmatrix}
 				\begin{bsmallmatrix}
 					0 \\-x^3 \\
 					x^2
 				\end{bsmallmatrix}
 				- \begin{bsmallmatrix}
 					0		&		0		&		0		\\
 					0		&	0		&	-1		\\
 					0		&	1	&	0
 				\end{bsmallmatrix} \begin{bsmallmatrix}
 					x^3 \\0 \\ -x^1
 				\end{bsmallmatrix} \\
 				&= \begin{bsmallmatrix}
 					x^2 \\ -x^1 \\ 0
 				\end{bsmallmatrix}\\
 				&= x^2 \partial_1 - x^1 \partial_2  \\
 				&= -z
 			\end{align*}
 			From the property $[X,Y] = -[Y,X]$, it follows that $[w,v] = z$. Similarly,
 			\begin{align*}
 				[v,z] &= J_z v - J_v z \\
 				&= \begin{bsmallmatrix}
 					0	&	-1	&	0 \\
 					1	&	0	&	0 \\
 					0	&	0	&	0	
 				\end{bsmallmatrix}
 				\begin{bsmallmatrix}
 					0 \\-x^3 \\
 					x^2
 				\end{bsmallmatrix}
 				- \begin{bsmallmatrix}
 					0		&		0		&		0		\\
 					0		&	0		&	-1		\\
 					0		&	1	&	0
 				\end{bsmallmatrix} \begin{bsmallmatrix}
 					-x^2 \\x^1 \\ 0
 				\end{bsmallmatrix} \\
 				&= \begin{bsmallmatrix}
 					x^3 \\ 0 \\ -x^1
 				\end{bsmallmatrix}\\
 				&= x^3 \partial_1 - x^1 \partial_3  \\
 				&= w
 			\end{align*}
 		
 			Again, $[z,v] = -w$. Finally,
 			\begin{align*}
 				[w,z] &= J_z w - J_w z \\
 				&= \begin{bsmallmatrix}
 					0	&	-1	&	0 \\
 					1	&	0	&	0 \\
 					0	&	0	&	0	
 				\end{bsmallmatrix}
 				\begin{bsmallmatrix}
 					x^3 \\0 \\
 					-x^1
 				\end{bsmallmatrix}
 				- \begin{bsmallmatrix}
 					0		&	0		&		1		\\
 					0		&	0		&		0		\\
 					-1		&	0		&		0
 				\end{bsmallmatrix} \begin{bsmallmatrix}
 					x^3 \\0 \\ -x^1
 				\end{bsmallmatrix} \\
 				&= \begin{bsmallmatrix}
 					-x^2 \\ x^1 \\ 0
 				\end{bsmallmatrix}\\
 				&= x^3 \partial_2 - x^2 \partial_3  \\
 				&= -v
 			\end{align*}
 			So in conclusion
 			\begin{subequations}
 				\begin{empheq}[box = \widefbox]{align}
 					[v,w] &= -z = -[w,v] \nonumber\\
 					[v,z] &= w = -[z,v] \nonumber\\
 					[w,z] &= -v = -[z,w] \nonumber
 				\end{empheq}
 			\end{subequations}
 		\pagebreak 
 		\appendix
 		\centering
 		$\quad $\\
 		\vspace{8cm}
 		\Huge $\ex$ \LARGE UwU A tensor transforms like a tensor UwU\footnote{To use this awesome Emoji package to make physics easier and clearer, as I have clearly done, simply define a command via \textbackslash newfontfamily\textbackslash SegoeEmoji$\{\.$Segoe UI Emoji$\.\}$. Proceed to use '\textbackslash DeclareMathOperator$\{\.$\textbackslash$<$keyword$>$$\.\}$$\{\.$SegoeEmoji $<$emoji$>$$\.\}$' to declare your variable of choice! Use in mathmode and encapsulate in brackets for sub/superscripts to avoid errors.}, \Huge $\ex$
 		\vfill
	 	
\end{document}